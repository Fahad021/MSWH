%% Generated by Sphinx.
\def\sphinxdocclass{report}
\documentclass[letterpaper,10pt,english,openany]{sphinxmanual}
\ifdefined\pdfpxdimen
   \let\sphinxpxdimen\pdfpxdimen\else\newdimen\sphinxpxdimen
\fi \sphinxpxdimen=.75bp\relax

\PassOptionsToPackage{warn}{textcomp}
\usepackage[utf8]{inputenc}
\ifdefined\DeclareUnicodeCharacter
% support both utf8 and utf8x syntaxes
  \ifdefined\DeclareUnicodeCharacterAsOptional
    \def\sphinxDUC#1{\DeclareUnicodeCharacter{"#1}}
  \else
    \let\sphinxDUC\DeclareUnicodeCharacter
  \fi
  \sphinxDUC{00A0}{\nobreakspace}
  \sphinxDUC{2500}{\sphinxunichar{2500}}
  \sphinxDUC{2502}{\sphinxunichar{2502}}
  \sphinxDUC{2514}{\sphinxunichar{2514}}
  \sphinxDUC{251C}{\sphinxunichar{251C}}
  \sphinxDUC{2572}{\textbackslash}
\fi
\usepackage{cmap}
\usepackage[T1]{fontenc}
\usepackage{amsmath,amssymb,amstext}
\usepackage{babel}



\usepackage{times}
\expandafter\ifx\csname T@LGR\endcsname\relax
\else
% LGR was declared as font encoding
  \substitutefont{LGR}{\rmdefault}{cmr}
  \substitutefont{LGR}{\sfdefault}{cmss}
  \substitutefont{LGR}{\ttdefault}{cmtt}
\fi
\expandafter\ifx\csname T@X2\endcsname\relax
  \expandafter\ifx\csname T@T2A\endcsname\relax
  \else
  % T2A was declared as font encoding
    \substitutefont{T2A}{\rmdefault}{cmr}
    \substitutefont{T2A}{\sfdefault}{cmss}
    \substitutefont{T2A}{\ttdefault}{cmtt}
  \fi
\else
% X2 was declared as font encoding
  \substitutefont{X2}{\rmdefault}{cmr}
  \substitutefont{X2}{\sfdefault}{cmss}
  \substitutefont{X2}{\ttdefault}{cmtt}
\fi


\usepackage[Bjarne]{fncychap}
\usepackage[,numfigreset=1,mathnumfig]{sphinx}

\fvset{fontsize=\small}
\usepackage{geometry}

% Include hyperref last.
\usepackage{hyperref}
% Fix anchor placement for figures with captions.
\usepackage{hypcap}% it must be loaded after hyperref.
% Set up styles of URL: it should be placed after hyperref.
\urlstyle{same}

\usepackage{sphinxmessages}
\setcounter{tocdepth}{2}



\title{Multiscale Solar Water Heating - Code Documentation}
\date{May 22, 2019}
\release{1.0}
\author{Milica Grahovac \\ Robert Hosbach \\ Hannes Gerhart \\ Katie Coughlin \\ Mohan Ganeshalingam \\ Vagelis Vossos \\ Hannes Gerhart}
\newcommand{\sphinxlogo}{\vbox{}}
\renewcommand{\releasename}{Release}
\makeindex
\begin{document}

\pagestyle{empty}
\sphinxmaketitle
\pagestyle{plain}
\sphinxtableofcontents
\pagestyle{normal}
\phantomsection\label{\detokenize{index::doc}}



\chapter{Multiscale Solar Water Heating (MSWH)}
\label{\detokenize{source/models:multiscale-solar-water-heating-mswh}}\label{\detokenize{source/models:sec-sys-mod}}\label{\detokenize{source/models::doc}}

\section{Scope}
\label{\detokenize{source/models:scope}}
The main purpose of the Multiscale Solar Water Heating (MSWH) software is to model energy use for individual and community scale solar water heating projects in California.

The package contains functional and unit tests and it is structured so that it can be extended with further technologies, applications and locations.


\section{Usage}
\label{\detokenize{source/models:usage}}
The user provides a climate zone for a project, an occupancy for each household and whether any of the occupants stay at home during the day. The software can then load a set of example California specific hourly domestic hot water end-use load profiles from a database, size and locate the systems. The user can now simulate the hourly system performance over a period of one representative year, visualize and explore the simulation results using time-series plots for temperature profiles, heat and power rates, or look at annual summaries. Similarly the user can model individual household solar water heating projects and base case conventional gas tank water heater systems, such that the results can be compared between the individual, community and base case systems.

This functionality is readily available through a Jupyter notebook and a Django web framework, depending on what level of detail the user would like to access. Please see the README file on the \sphinxhref{https://github.com/LBNL-ETA/MSWH}{MSWH repo} for usage and installation details.


\section{Features}
\label{\detokenize{source/models:features}}
This software package contains the following Python modules:
\begin{itemize}
\item {} 
Solar irradiation on a tilted surface

\item {} 
Simplified component models for Converter (solar collectors, electric resistance heater, gas burner, photovoltaic panels, heat pump), Storage (solar thermal tank, heat pump thermal tank, conventional gas tank water heater), and Distribution (distribution and solar pump, piping losses) components

\item {} 
Preconfigured system simulation models for: base case gas tank water heaters, solar thermal water heaters (solar collector feeding a storage tank, with a tankeless gas water heater backup in a new installation cases and a basecase gas tank water heater in a retrofit case) and solar electric water heaters (heat pump storage tank with an electric resistance backup)

\item {} 
Database with component performance parameters, California specific weather data and domestic hot water end-use load profiles

\item {} 
Django web framework to configure project, parametrize components and run simulation from a web browser

\end{itemize}


\section{Approach to System Modeling and Simulation}
\label{\detokenize{source/models:approach-to-system-modeling-and-simulation}}
The energy sources we consider are solar irradiation, gas and electricity. The source energy is converted, if needed stored, and distributed to meet the end-use loads for each household.

Upon assembling the components into systems, we perform an annual simulation with hourly timesteps. We solve any differential equations for each time step using an explicit forward Euler method, a first order technique that provides a good approximation given the dynamics of the process observed and the level of detail required in our analysis.

We configure and size each MSWH thermal configuration so that it complies with the CSI-thermal rebate program sizing requirements. The system model assumes appropriate flow and temperature controls and includes freeze and stagnation protection.


\chapter{Python Code Documentation}
\label{\detokenize{source/modules:python-code-documentation}}\label{\detokenize{source/modules::doc}}

\section{Subpackages}
\label{\detokenize{source/modules:subpackages}}

\subsection{System and Component Models}
\label{\detokenize{source/mswh.system:system-and-component-models}}\label{\detokenize{source/mswh.system::doc}}

\subsubsection{mswh.system.components module}
\label{\detokenize{source/mswh.system:module-mswh.system.components}}\label{\detokenize{source/mswh.system:mswh-system-components-module}}\index{mswh.system.components (module)@\spxentry{mswh.system.components}\spxextra{module}}\index{Converter (class in mswh.system.components)@\spxentry{Converter}\spxextra{class in mswh.system.components}}

\begin{fulllineitems}
\phantomsection\label{\detokenize{source/mswh.system:mswh.system.components.Converter}}\pysiglinewithargsret{\sphinxbfcode{\sphinxupquote{class }}\sphinxcode{\sphinxupquote{mswh.system.components.}}\sphinxbfcode{\sphinxupquote{Converter}}}{\emph{params=None}, \emph{weather=None}, \emph{sizes=1.0}, \emph{log\_level=10}}{}
Bases: \sphinxcode{\sphinxupquote{object}}

Contains energy converter models, such as
solar collectors, electric resistance heaters, gas burners,
photovoltaic panels, and heat pumps. Depending on the intended
usage, the models can be used to determine either a time period
of component operation (for example an entire year), or a single
timestep of component performance.

Parameters:
\begin{quote}
\begin{description}
\item[{params: pd df}] \leavevmode
Component performance parameters per project
Default: None (default model parameters will get used)

\item[{weather: pd df}] \leavevmode
Weather data timeseeries with columns: amb. temp,
solar irradiation. Number of rows equals the number of timesteps.
Default: None (constant values will be set - use for
a single timestep calculation, or if passing arguments
directly to static methods)

\item[{sizes: pd df}] \leavevmode
Component sizes per project.
Default: 1. (see individual components for specifics)

\item[{log\_level: None or python logger logging level,}] \leavevmode
Default: logging.DEBUG
This applies for a subset of the class functionality, mostly
used to deprecate logger messages for certain calculations.
For Example: log\_level = logging.ERROR will only throw error
messages and ignore INFO, DEBUG and WARNING.

\end{description}
\end{quote}

Note:
\begin{quote}

If more than one of the same component is a part of the
system, a separate instance of the converter should
be created for each instance of the component.

Each component is also implemented as a static method that
can be used outside of this framework.
\end{quote}

Examples:
\begin{quote}

See \sphinxcode{\sphinxupquote{{}`swh.system.tests.test\_components{}`}} module and
\sphinxcode{\sphinxupquote{{}`scripts/Project Level SWH System Tool.ipynb{}`}}
for examples on how to use the methods as stand alone and
in a system model simulation.
\end{quote}
\index{electric\_resistance() (mswh.system.components.Converter method)@\spxentry{electric\_resistance()}\spxextra{mswh.system.components.Converter method}}

\begin{fulllineitems}
\phantomsection\label{\detokenize{source/mswh.system:mswh.system.components.Converter.electric_resistance}}\pysiglinewithargsret{\sphinxbfcode{\sphinxupquote{electric\_resistance}}}{\emph{Q\_dem}}{}
Electric resistance heater model. Can be
used both as an instantaneous electric WH and as
an auxiliary heater within the thermal tank.

Parameters:
\begin{quote}
\begin{description}
\item[{Q\_dem: float or array like, {[}W{]}}] \leavevmode
Heat demand

\end{description}
\end{quote}

Returns:
\begin{quote}
\begin{description}
\item[{res: dict}] \leavevmode
self.r{[}‘q\_del\_bckp’{]} : float, array - delivered heat rate, {[}W{]}
self.r{[}‘q\_el\_use’{]} : float, array - electricity use, {[}W{]}
self.r{[}‘q\_unmet’{]} : float, array - unmet demand heat rate, {[}W{]}

\end{description}
\end{quote}

\end{fulllineitems}

\index{gas\_burner() (mswh.system.components.Converter method)@\spxentry{gas\_burner()}\spxextra{mswh.system.components.Converter method}}

\begin{fulllineitems}
\phantomsection\label{\detokenize{source/mswh.system:mswh.system.components.Converter.gas_burner}}\pysiglinewithargsret{\sphinxbfcode{\sphinxupquote{gas\_burner}}}{\emph{Q\_dem}}{}
Gas burner model. Used both
as an instantaneous gas WH and as a
gas backup for solar thermal.

Parameters:
\begin{quote}
\begin{description}
\item[{Q\_dem: float or array like, W}] \leavevmode
Heat demand

\end{description}
\end{quote}

Returns:
\begin{quote}
\begin{description}
\item[{res: dict}] \leavevmode
self.r{[}‘q\_del\_bckp’{]} : float, array - delivered heat rate, {[}W{]}
self.r{[}‘q\_gas\_use’{]} : float, array - gas use heat rate, {[}W{]}
self.r{[}‘q\_unmet’{]} : float, array - unmet demand heat rate, {[}W{]}

Any further unit conversion should be performed
using unit\_converters.Utility class

\end{description}
\end{quote}

\end{fulllineitems}

\index{heat\_pump() (mswh.system.components.Converter method)@\spxentry{heat\_pump()}\spxextra{mswh.system.components.Converter method}}

\begin{fulllineitems}
\phantomsection\label{\detokenize{source/mswh.system:mswh.system.components.Converter.heat_pump}}\pysiglinewithargsret{\sphinxbfcode{\sphinxupquote{heat\_pump}}}{\emph{T\_wet\_bulb}, \emph{T\_tank}}{}
Returns the current heating performance and electricity usage
in the current conditions depending on wet bulb temperature,
average tank water temperature, and the rated heating performance.

Rated conditions are: wet bulb = 14 degC, tank = 48.9 degC

Parameters:
\begin{quote}
\begin{description}
\item[{T\_wet\_bulb: real, array}] \leavevmode
Inlet air wet bulb temperature {[}K{]}

\item[{T\_tank: real, array}] \leavevmode
Water temperature in the storage tank {[}K{]}

\item[{C1: real}] \leavevmode
Coefficient 1, either for normalized COP or heating
capacity curve {[}-{]}

\item[{C2: real}] \leavevmode
Coefficient 2, either for normalized COP or heating
capacity curve {[}1/degC{]}

\item[{C3: real}] \leavevmode
Coefficient 3, either for normalized COP or heating
capacity curve {[}1/degC2{]}

\item[{C4: real}] \leavevmode
Coefficient 4, either for normalized COP or heating
capacity curve {[}1/degC{]}

\item[{C5: real}] \leavevmode
Coefficient 5, either for normalized COP or heating
capacity curve {[}1/degC2{]}

\item[{C6: real}] \leavevmode
Coefficient 6, either for normalized COP or heating
capacity curve {[}1/degC2{]}

\end{description}
\end{quote}

Returns:
\begin{quote}
\begin{description}
\item[{performance: dict}] \leavevmode
\{‘cop’: current Coefficient Of Performance (COP) of
heat pump, {[}-{]}
\begin{quote}

‘heat\_cap’: current heating capacity of heat pump, {[}W{]}
‘el\_use’: current electricity use of heat pump
(heat\_cap / cop), {[}W{]}\}
\end{quote}

\end{description}
\end{quote}

\end{fulllineitems}

\index{photovoltaic() (mswh.system.components.Converter method)@\spxentry{photovoltaic()}\spxextra{mswh.system.components.Converter method}}

\begin{fulllineitems}
\phantomsection\label{\detokenize{source/mswh.system:mswh.system.components.Converter.photovoltaic}}\pysiglinewithargsret{\sphinxbfcode{\sphinxupquote{photovoltaic}}}{\emph{use\_p\_peak=True}, \emph{inc\_rad=None}}{}
Photovoltaic model

Parameters:
\begin{quote}
\begin{description}
\item[{use\_p\_peak: boolean}] \leavevmode
Boolean flag determining if peak power is used for sizing
the pv panel (instead of area and efficiency)

\end{description}
\end{quote}

Returns:
\begin{quote}
\begin{description}
\item[{self.pv\_power: dict of floats}] \leavevmode
Generated power {[}W{]}
‘ac’ : AC
‘dc’ : DC

\end{description}
\end{quote}

\end{fulllineitems}

\index{size (mswh.system.components.Converter attribute)@\spxentry{size}\spxextra{mswh.system.components.Converter attribute}}

\begin{fulllineitems}
\phantomsection\label{\detokenize{source/mswh.system:mswh.system.components.Converter.size}}\pysigline{\sphinxbfcode{\sphinxupquote{size}}}
\end{fulllineitems}

\index{solar\_collector() (mswh.system.components.Converter method)@\spxentry{solar\_collector()}\spxextra{mswh.system.components.Converter method}}

\begin{fulllineitems}
\phantomsection\label{\detokenize{source/mswh.system:mswh.system.components.Converter.solar_collector}}\pysiglinewithargsret{\sphinxbfcode{\sphinxupquote{solar\_collector}}}{\emph{t\_in}, \emph{t\_amb=None}, \emph{inc\_rad=None}}{}
Two commonly used empirical instantaneous collector
efficiency models based on test data from standard
test procedures (SRCC, ISO9806), found in
J. A. Duffie and W. A. Beckman, Solar engineering of thermal processes, 3rd ed. Hoboken, N.J: Wiley, 2006., are:
\begin{itemize}
\item {} 
Cooper and Dunkle (CD model, eq 6.17.7)

\item {} 
Hottel-Whillier-Bliss (HWB model, eq 6.16.1, 6.7.6)

\end{itemize}

Parameters:
\begin{quote}
\begin{description}
\item[{t\_in: float, array}] \leavevmode
Collector inlet temperature (timeseries) {[}K{]}

\item[{t\_amb: float, array}] \leavevmode
Ambient temperature (timeseries) {[}K{]}
Default: None (to use data extracted from the weather df)

\item[{inc\_rad: float, array}] \leavevmode
Incident radiation (timeseries) {[}W{]}
Default: None (to use data extracted from the weather df)

\end{description}
\end{quote}

Returns:
\begin{quote}

res: dict or floats or arrays
\begin{quote}
\begin{description}
\item[{\{‘Q\_gain’}] \leavevmode{[}Solar gains from the gross collector area, {[}W{]}{]}
‘eff’ : Efficiency of solar to heat conversion, {[}-{]}

\end{description}
\end{quote}
\end{quote}

\end{fulllineitems}

\index{weather (mswh.system.components.Converter attribute)@\spxentry{weather}\spxextra{mswh.system.components.Converter attribute}}

\begin{fulllineitems}
\phantomsection\label{\detokenize{source/mswh.system:mswh.system.components.Converter.weather}}\pysigline{\sphinxbfcode{\sphinxupquote{weather}}}
\end{fulllineitems}


\end{fulllineitems}

\index{Distribution (class in mswh.system.components)@\spxentry{Distribution}\spxextra{class in mswh.system.components}}

\begin{fulllineitems}
\phantomsection\label{\detokenize{source/mswh.system:mswh.system.components.Distribution}}\pysiglinewithargsret{\sphinxbfcode{\sphinxupquote{class }}\sphinxcode{\sphinxupquote{mswh.system.components.}}\sphinxbfcode{\sphinxupquote{Distribution}}}{\emph{params=None}, \emph{sizes=1.0}, \emph{fluid\_medium='water'}, \emph{timestep=1.0}, \emph{log\_level=10}}{}
Bases: \sphinxcode{\sphinxupquote{object}}

Describes performance of distribution system components.
\begin{description}
\item[{Parameters:}] \leavevmode\begin{description}
\item[{sizes: pd df}] \leavevmode
Pandas dataframe with component sizes, or 1.

\item[{fluid\_medium: string}] \leavevmode
Default: ‘water’. No other options implemented

\item[{timestep: float, h}] \leavevmode
Duration of a single timestep, in hours, defaults to 1.

\item[{log\_level: None or python logger logging level,}] \leavevmode
Default: logging.DEBUG
This applies for a subset of the class functionality, mostly
used to deprecate logger messages for certain calculations.
For Example: log\_level = logging.ERROR will only throw error
messages and ignore INFO, DEBUG and WARNING.

\end{description}

\end{description}

Note:
\begin{quote}

Each component is also implemented as a static method that
can be used outside of this framework.
\end{quote}

Examples:
\begin{quote}

See \sphinxcode{\sphinxupquote{{}`swh.system.tests.test\_components{}`}} module and
for examples on how to use the methods.
\end{quote}
\index{pipe\_losses() (mswh.system.components.Distribution method)@\spxentry{pipe\_losses()}\spxextra{mswh.system.components.Distribution method}}

\begin{fulllineitems}
\phantomsection\label{\detokenize{source/mswh.system:mswh.system.components.Distribution.pipe_losses}}\pysiglinewithargsret{\sphinxbfcode{\sphinxupquote{pipe\_losses}}}{\emph{T\_in=333.15}, \emph{T\_amb=293.15}, \emph{V\_tap=0.05}, \emph{max\_V\_tap=0.1514}}{}
Thermal losses from distribution pipes.

Parameters:
\begin{quote}
\begin{description}
\item[{T\_in: float, K}] \leavevmode
Hot water temperature at distribution pipe inlet

\item[{T\_amb: float, K}] \leavevmode
Ambient temperature

\item[{V\_tap: float, m3/h}] \leavevmode
Timestep draw volume

\item[{max\_V\_tap: float, m3/h}] \leavevmode
Maximum draw volume, m3/h (design variable)

\end{description}
\end{quote}

Returns:
\begin{quote}
\begin{description}
\item[{res: dict}] \leavevmode
{[}‘heat\_rate’{]}: Loss heat rate, W

\end{description}
\end{quote}

\end{fulllineitems}

\index{pump() (mswh.system.components.Distribution method)@\spxentry{pump()}\spxextra{mswh.system.components.Distribution method}}

\begin{fulllineitems}
\phantomsection\label{\detokenize{source/mswh.system:mswh.system.components.Distribution.pump}}\pysiglinewithargsret{\sphinxbfcode{\sphinxupquote{pump}}}{\emph{on\_array=array({[}1.}, \emph{1.}, \emph{1.}, \emph{...}, \emph{1.}, \emph{1.}, \emph{1.{]})}, \emph{role='solar'}}{}
Solar and distribution pump energy use.
Assumes a fixed speed pump.

Parameters:
\begin{quote}
\begin{description}
\item[{on\_array: array}] \leavevmode
Pump on/off status for the chosen number of discrete
timesteps
Default: np.ones(8760) - on for a year in hourly timesteps.

\item[{role: string}] \leavevmode
‘solar’ : primary (solar collector) loop
‘distribution’ : secondary (distribution) loop

\end{description}
\end{quote}

Returns:
\begin{quote}

en\_use: float or array like
\end{quote}

\end{fulllineitems}

\index{size (mswh.system.components.Distribution attribute)@\spxentry{size}\spxextra{mswh.system.components.Distribution attribute}}

\begin{fulllineitems}
\phantomsection\label{\detokenize{source/mswh.system:mswh.system.components.Distribution.size}}\pysigline{\sphinxbfcode{\sphinxupquote{size}}}
\end{fulllineitems}


\end{fulllineitems}

\index{Storage (class in mswh.system.components)@\spxentry{Storage}\spxextra{class in mswh.system.components}}

\begin{fulllineitems}
\phantomsection\label{\detokenize{source/mswh.system:mswh.system.components.Storage}}\pysiglinewithargsret{\sphinxbfcode{\sphinxupquote{class }}\sphinxcode{\sphinxupquote{mswh.system.components.}}\sphinxbfcode{\sphinxupquote{Storage}}}{\emph{params=None}, \emph{size=1.0}, \emph{type='sol\_tank'}, \emph{timestep=1.0}, \emph{log\_level=10}}{}
Bases: \sphinxcode{\sphinxupquote{object}}

Describes performance of storage components, such as
solar thermal tank, heat pump thermal tank, conventional gas
tank water heater.

Parameters:
\begin{quote}
\begin{description}
\item[{params: pd df}] \leavevmode
Component performance parameters per project
Default: None. See tests and examples on how to
structure this input.

\item[{weather: pd df}] \leavevmode
Weather data timeseeries (amb. temp, solar irradiation)
Default: None. See tests and examples on how to
structure this input.

\item[{size: pd df or float, m3}] \leavevmode
Tank size.
Default 1. See tests and examples on how to
structure this input.

\item[{type: string}] \leavevmode
Type of storage component. Options:
‘sol\_tank’ - indirect tank WH with a coil to circulate
fluid heated by a solar collector
‘hp\_tank’ - tank with an inbuilt heat pump
‘wham\_tank’ - conventional gas tank water heater model
based on a WH model from the efficiency standards analysis
‘gas\_tank’ - conventional gas tank water heater ({\color{red}\bfseries{}*}mg not
implemented)

\item[{log\_level: None or python logger logging level,}] \leavevmode
Default: logging.DEBUG
This applies for a subset of the class functionality, mostly
used to deprecate logger messages for certain calculations.
For Example: log\_level = logging.ERROR will only throw error
messages and ignore INFO, DEBUG and WARNING.

\item[{timestep: float, h}] \leavevmode
Duration of a single timestep, in hours, defaults to 1.

\end{description}
\end{quote}

Note:
\begin{quote}

Create a new instance of the class for each storage component.
\end{quote}

Examples:
\begin{quote}

See \sphinxcode{\sphinxupquote{{}`swh.system.tests.test\_components{}`}} module and
\sphinxcode{\sphinxupquote{{}`scripts/Project Level SWH System Tool.ipynb{}`}}
for examples on how to use the methods as stand alone and
in a system model simulation.
\end{quote}
\index{electric\_tank\_wh() (mswh.system.components.Storage method)@\spxentry{electric\_tank\_wh()}\spxextra{mswh.system.components.Storage method}}

\begin{fulllineitems}
\phantomsection\label{\detokenize{source/mswh.system:mswh.system.components.Storage.electric_tank_wh}}\pysiglinewithargsret{\sphinxbfcode{\sphinxupquote{electric\_tank\_wh}}}{}{}
Currently not implemented.

\end{fulllineitems}

\index{gas\_tank\_wh() (mswh.system.components.Storage method)@\spxentry{gas\_tank\_wh()}\spxextra{mswh.system.components.Storage method}}

\begin{fulllineitems}
\phantomsection\label{\detokenize{source/mswh.system:mswh.system.components.Storage.gas_tank_wh}}\pysiglinewithargsret{\sphinxbfcode{\sphinxupquote{gas\_tank\_wh}}}{\emph{V\_draw}, \emph{T\_feed}, \emph{T\_amb=291.48}}{}
Gas tank model for state-level analysis based on WHAM equation:
\begin{description}
\item[{Q\_dot\_cons {[}W{]} =}] \leavevmode\begin{description}
\item[{(V\_dot\_draw * rho * c * (T\_draw,set - T\_feed)) /}] \leavevmode
n\_re * (1-(U*A*(T\_draw,set - T\_amb))/P\_rated) +

\end{description}

U*A*(T\_draw,set - T\_amb)

\end{description}

Parameters:
\begin{quote}
\begin{description}
\item[{V\_draw: float or array like, m3/h}] \leavevmode
Hourly water draw for a single timestep of
an entire analysis period

\item[{T\_feed: float or array like, K}] \leavevmode
Temperature of water heater inlet water for
a single timestep of an entire analysis period

\item[{T\_amb: float or array like, K}] \leavevmode
Temperature of space surrounding water heater
Default: 65 degF

\end{description}
\end{quote}

Returns:
\begin{quote}
\begin{description}
\item[{res: dict}] \leavevmode
self.r{[}‘q\_del’{]} : float, array - delivered heat rate, {[}W{]}
self.r{[}‘q\_dem’{]} : float, array - demand heat rate, {[}W{]}
self.r{[}‘q\_gas\_use’{]} : float, array - gas use heat rate, {[}W{]}
self.r{[}‘q\_unmet’{]} : float, array - unmet demand, {[}w{]}
self.r{[}‘q\_dump’{]} : float, array - dumped heat, {[}W{]}

\end{description}
\end{quote}

Note:
\begin{quote}

Assuming no electricity consumption in this version.

Make sure to size the tank according to the recommended
sizing rules, since the WHAM model does not apply to
tanks that are not appropriately sized.
\end{quote}

\end{fulllineitems}

\index{setup\_electric() (mswh.system.components.Storage method)@\spxentry{setup\_electric()}\spxextra{mswh.system.components.Storage method}}

\begin{fulllineitems}
\phantomsection\label{\detokenize{source/mswh.system:mswh.system.components.Storage.setup_electric}}\pysiglinewithargsret{\sphinxbfcode{\sphinxupquote{setup\_electric}}}{}{}
Currently not implemented.

\end{fulllineitems}

\index{setup\_thermal() (mswh.system.components.Storage method)@\spxentry{setup\_thermal()}\spxextra{mswh.system.components.Storage method}}

\begin{fulllineitems}
\phantomsection\label{\detokenize{source/mswh.system:mswh.system.components.Storage.setup_thermal}}\pysiglinewithargsret{\sphinxbfcode{\sphinxupquote{setup\_thermal}}}{\emph{medium='water'}, \emph{split\_tank=True}, \emph{vol\_fra\_upper=0.5}, \emph{h\_vs\_r=6.0}, \emph{dT\_param=2.0}, \emph{T\_max=344.15}, \emph{T\_draw\_set=322.04}, \emph{insul\_thickness=0.085}, \emph{spec\_hea\_cond=0.04}, \emph{coil\_eff=0.84}, \emph{tank\_re=0.76}, \emph{dT\_err\_max=2.0}, \emph{gas\_heater\_autosize=False}}{}
Sets thermal storage variables related to:
\begin{itemize}
\item {} 
loss calculation

\item {} 
distribution of net gains/losses within two
tank volumes (upper and lower)

\end{itemize}

Parameters:
\begin{quote}
\begin{description}
\item[{medimum: string}] \leavevmode
Storing medium (for thermal defaults to ‘water’)

\item[{vol\_fra\_upper: float}] \leavevmode
Fraction of storage volume assigned to the upper
tank volume (applies to ‘thermal’ only)
If split\_tank set to False, the value is ignored

\item[{dT\_param: float, K}] \leavevmode
Used as:
\begin{itemize}
\item {} 
Maximum temperature difference expected to occur

\end{itemize}

between the upper and the lower tank volume while charging
\begin{itemize}
\item {} 
In-tank-coil approach

\end{itemize}

\item[{h\_vs\_r: float}] \leavevmode
Regression parameter - tank height/diameter ratio
(based on web scraped data, see
\sphinxtitleref{scripts/Gallons vs. Height} notebook), default: 6.

\item[{T\_max: float, K}] \leavevmode
Maximum allowed fluid temperature in the thermal
storage tank, defaults to 344.15 K = 71 degC.

\item[{T\_draw\_set: float, K}] \leavevmode
Draw temperature used in the
load calculation, defaults to
120 degF = 322.04 K = 48.89 degC

\item[{coil\_eff: float}] \leavevmode
Simplified efficiency of the coil heat exchanger
Used in modeling of indirect coil-in-tank water heaters
It excludes the approach temperature and represents
the remaining heat transfer inefficiency

\item[{tank\_re: float}] \leavevmode
Recovery efficiency of a gas tank water heater.
Used for the Storage.gas\_tank\_wh model

\item[{dT\_err\_max: float}] \leavevmode
Allowed dT error below the minimum
tank temperature due to finite timestep length
approximation

\item[{gas\_heater\_autosize: boolean}] \leavevmode
There is a gas heater in the tank and it will be
autosized based on the tank volume

\end{description}
\end{quote}

\end{fulllineitems}

\index{size (mswh.system.components.Storage attribute)@\spxentry{size}\spxextra{mswh.system.components.Storage attribute}}

\begin{fulllineitems}
\phantomsection\label{\detokenize{source/mswh.system:mswh.system.components.Storage.size}}\pysigline{\sphinxbfcode{\sphinxupquote{size}}}
\end{fulllineitems}

\index{tap() (mswh.system.components.Storage method)@\spxentry{tap()}\spxextra{mswh.system.components.Storage method}}

\begin{fulllineitems}
\phantomsection\label{\detokenize{source/mswh.system:mswh.system.components.Storage.tap}}\pysiglinewithargsret{\sphinxbfcode{\sphinxupquote{tap}}}{\emph{V\_draw\_load}, \emph{T\_tank}, \emph{T\_feed}, \emph{dT\_loss=0.0}, \emph{T\_draw\_min=None}}{}
Calculates the water draw volume and
heat content drawn from the top of an infinitely
large adiabatic tank given the hot water demand,
tank temperature and the water main temperature.

It functions somewhat similarly to a
thermostatic valve since it regulates the
tap flow from the tank as follows:
\begin{quote}
\begin{itemize}
\item {} 
limits above if the tank temperature

\end{itemize}

is higher than the nominal draw temperature
\begin{itemize}
\item {} 
tap flow equals V\_draw\_load for any

\end{itemize}

tank temperature between T\_draw\_min
and T\_draw\_nom
\begin{itemize}
\item {} 
tap flow is zero if tank temperature

\end{itemize}

is below T\_draw\_min and T\_draw\_min is
provided
\end{quote}

The results represent the theoretical limit
for the draw. The tank model will check if the
full amount can be delivered or only a
part of the demand, due to the limited
tank volume and thermal losses from the tank,
and adjust the values.

Parameters:
\begin{quote}
\begin{description}
\item[{V\_draw\_load: float, m3/h}] \leavevmode
Volume of DHW drawn at the nominal
end-use load temperature.

\item[{T\_tank: float, K}] \leavevmode
Tank node temperature from which the
DHW is being tapped (usually the
upper volume)

\item[{T\_feed: float or array, K}] \leavevmode
Temperature of water heater inlet water

\item[{dT\_loss: float, K}] \leavevmode
Distribution loss temperature difference

\item[{T\_draw\_min: float, K}] \leavevmode
Minimal temperature that needs to
be achieved in the tank in order
to allow tapping.
Default: None - tapping is always
enabled
Recommended usage - in colder climates
where an outdoors tank may be cooler
than the water main.

\end{description}
\end{quote}

Returns:
\begin{quote}
\begin{description}
\item[{draw: dict}] \leavevmode
Draw volume: ‘vol’, m3/h
Total demand heat rate: ‘tot\_dem’, W
Infinite volume delivered heat rate: ‘heat\_rate’, W
Infinite volume unmet heat rate: ‘unmet\_heat\_rate’, W

\end{description}
\end{quote}

\end{fulllineitems}

\index{thermal\_tank() (mswh.system.components.Storage method)@\spxentry{thermal\_tank()}\spxextra{mswh.system.components.Storage method}}

\begin{fulllineitems}
\phantomsection\label{\detokenize{source/mswh.system:mswh.system.components.Storage.thermal_tank}}\pysiglinewithargsret{\sphinxbfcode{\sphinxupquote{thermal\_tank}}}{\emph{pre\_T\_amb=293.15}, \emph{pre\_T\_feed=291.15}, \emph{pre\_T\_upper=328.15}, \emph{pre\_T\_lower=323.15}, \emph{pre\_V\_tap=0.00757}, \emph{pre\_Q\_in=400.0}, \emph{max\_V\_tap=0.1514}}{}
Model of a thermal storage tank with:
\begin{itemize}
\item {} 
coil heat exchanger for the solar gains

\item {} 
DHW tap at the top of the tank

\item {} 
recharge tap at the bottom of the tank

\end{itemize}

The model can be instantiated as a:
\begin{itemize}
\item {} 
solar thermal tank

\item {} 
heat pump tank

\end{itemize}

Parameters:
\begin{quote}
\begin{description}
\item[{type: string}] \leavevmode
‘solar’ - solar tank (assumes
that heated fluid from a solar collector is circulated
through an in-tank-coil)
‘hp’ - heat pump tank (assumes an inbuilt heat pump
as a main heat source)

The type will affect output labeling and heat transfer
efficiency.

\item[{pre\_T\_amb: float, K}] \leavevmode
Ambient temperature

\item[{pre\_T\_feed: float, K}] \leavevmode
Temperature of the water
that replenishes the tapped volume
(e.g. water main temperature)

\item[{pre\_T\_upper: float, K}] \leavevmode
Upper tank volume temperature

\item[{pre\_T\_lower: float, K}] \leavevmode
Lower tank volume temperature

\item[{pre\_Q\_in: float, W}] \leavevmode
Heat gain passed to in-tank coil from solar collector
or from a heat pump, depending on the type

\item[{pre\_V\_tap: float, m3/h}] \leavevmode
Volume of water tapped from the top of the tank

\item[{max\_V\_tap: float, m3/h}] \leavevmode
Annual peak flow

\end{description}
\end{quote}

Returns:
\begin{quote}
\begin{description}
\item[{res: dict}] \leavevmode\begin{quote}

Single Timestep input and output values for temperatures {[}K{]}
and heat rates {[}W{]}:
\end{quote}
\begin{description}
\item[{\{net\_gain\_label}] \leavevmode{[}pre\_Q\_in\_net,                self.r{[}‘q\_loss\_low’{]}{]}{[}pre\_Q\_loss\_lower,                self.r{[}‘q\_loss\_up’{]}{]}{[}pre\_Q\_loss\_upper,                \# demand, delivered and unmet heat{]}
\# (between tap setpoint and water main)
self.r{[}‘q\_dem’{]} : tap{[}‘net\_dem’{]},                self.r{[}‘q\_dem\_tot’{]} : tap{[}‘tot\_dem’{]},                self.r{[}‘q\_del\_tank’{]} : tank{[}self.r{[}‘q\_del\_tank’{]}{]},                self.r{[}‘q\_unmet\_tank’{]} :                     np.round(                tank{[}self.r{[}‘q\_unmet\_tank’{]}{]} + tap{[}‘unmet\_heat\_rate’{]}, 2),                self.r{[}‘q\_dump’{]} : tank{[}self.r{[}‘q\_dump’{]}{]},                self.r{[}‘q\_ovrcool\_tank’{]} : tank{[}self.r{[}‘q\_ovrcool\_tank’{]}{]},                self.r{[}‘q\_dem\_balance’{]} : np.round(Q\_dem\_balance),                \# average temperatures for tank volumes
self.r{[}‘t\_tank\_low’{]} : tank{[}self.r{[}‘t\_tank\_low’{]}{]},                self.r{[}‘t\_tank\_up’{]} : tank{[}self.r{[}‘t\_tank\_up’{]}{]},                self.r{[}‘dt\_dist’{]} : dist{[}‘dt\_dist’{]},                self.r{[}‘t\_set’{]} : self.T\_draw\_set,                self.r{[}‘q\_dist\_loss’{]} : dist{[}‘heat\_loss’{]},
self.r{[}‘flow\_on\_frac’{]} : dist{[}‘flow\_on\_frac’{]}\}

Temperatures in K, heat rates in W

\end{description}

\end{description}
\end{quote}

\end{fulllineitems}

\index{thermal\_tank\_dynamics() (mswh.system.components.Storage method)@\spxentry{thermal\_tank\_dynamics()}\spxextra{mswh.system.components.Storage method}}

\begin{fulllineitems}
\phantomsection\label{\detokenize{source/mswh.system:mswh.system.components.Storage.thermal_tank_dynamics}}\pysiglinewithargsret{\sphinxbfcode{\sphinxupquote{thermal\_tank\_dynamics}}}{\emph{pre\_T\_amb}, \emph{pre\_T\_upper}, \emph{pre\_T\_lower}, \emph{pre\_Q\_in}, \emph{pre\_Q\_loss\_upper}, \emph{pre\_Q\_loss\_lower}, \emph{pre\_T\_feed}, \emph{pre\_Q\_tap}}{}
Partial model of a thermal storage tank.
Applies first order forward marching Euler method and updates
the tank state for the current timestep
based on the enthalpy balance and simplified
assumptions about stratification. Thus, all
input variables pertain to the previous timestep,
while the outputs are solutions for the current timestep.

For example partial model application see thermal\_tank method.

See inline comments for detailed explanation of the model.

Parameters:
\begin{quote}
\begin{description}
\item[{pre\_T\_amb: float, K}] \leavevmode
Ambient air temperature

\item[{pre\_T\_upper: float, K}] \leavevmode
Upper tank volume temperature

\item[{pre\_T\_lower: float, K}] \leavevmode
Lower tank volume temperature

It is recommended to set equal initial values
for pre\_T\_upper and pre\_T\_lower

\item[{pre\_Q\_in: float, W}] \leavevmode
Total heat gain (e.g. from a coil heat exchanger,
a heating element, etc.)

\item[{pre\_Q\_loss\_upper: float, W}] \leavevmode
Heat loss from the upper tank volume

\item[{pre\_T\_lower: float, W}] \leavevmode
Heat loss from the lower tank volume

\item[{pre\_T\_feed: float, K}] \leavevmode
Temperature of the water
that replenishes the tapped volume
(e.g. water main temperature)

\item[{pre\_Q\_tap: float, W}] \leavevmode
Heat loss that would occur if the tank
volume at pre\_T\_upper was infinite

\end{description}
\end{quote}

Returns:
\begin{quote}
\begin{description}
\item[{res: dict of floats}] \leavevmode
Represent averages in a single timestep.
Average temperatures for tank volumes:
self.r{[}self.r{[}‘t\_tank\_low’{]}{]} : lower, K
self.r{[}‘t\_tank\_up’{]} : upper, K
Heat rates:
‘Q\_net’ : expected timestep net gain/loss based on inputs, W
self.r{[}‘q\_dump’{]} : dumped heat, W
‘Q\_draw’ : delivered to load W
‘Q\_draw\_unmet’ : unmet load due to finite tank volume, W
self.r{[}‘q\_ovrcool\_tank’{]} : error in balancing due to minimal
tank temperature limit assumption in each timestep

Note: ‘Q\_draw’ + ‘Q\_draw\_unmet’ = pre\_Q\_tap

\end{description}
\end{quote}

\end{fulllineitems}

\index{volume\_to\_power() (mswh.system.components.Storage method)@\spxentry{volume\_to\_power()}\spxextra{mswh.system.components.Storage method}}

\begin{fulllineitems}
\phantomsection\label{\detokenize{source/mswh.system:mswh.system.components.Storage.volume_to_power}}\pysiglinewithargsret{\sphinxbfcode{\sphinxupquote{volume\_to\_power}}}{\emph{tank\_volume}}{}
Method to convert a gas water heater’s volume input power
based on a linear regression of Prospector data.
Look in the X drive Data/Water Heaters/Regressions folder
for the WaterHeater\_ScrapeData\_Python.xlsx file.
Parameters:
\begin{quote}
\begin{description}
\item[{tank\_volume: float or int}] \leavevmode
Water heater tank volume {[}m3{]}

\end{description}
\end{quote}

Returns
\begin{quote}
\begin{description}
\item[{tank\_input\_power: float}] \leavevmode
Water heater input (rated) power {[}W{]}

\end{description}
\end{quote}

\end{fulllineitems}


\end{fulllineitems}



\subsubsection{mswh.system.models module}
\label{\detokenize{source/mswh.system:module-mswh.system.models}}\label{\detokenize{source/mswh.system:mswh-system-models-module}}\index{mswh.system.models (module)@\spxentry{mswh.system.models}\spxextra{module}}\index{System (class in mswh.system.models)@\spxentry{System}\spxextra{class in mswh.system.models}}

\begin{fulllineitems}
\phantomsection\label{\detokenize{source/mswh.system:mswh.system.models.System}}\pysiglinewithargsret{\sphinxbfcode{\sphinxupquote{class }}\sphinxcode{\sphinxupquote{mswh.system.models.}}\sphinxbfcode{\sphinxupquote{System}}}{\emph{sys\_params=None}, \emph{backup\_params=None}, \emph{weather=None}, \emph{sys\_sizes=1.0}, \emph{backup\_sizes=1.0}, \emph{loads=None}, \emph{timestep=1.0}, \emph{log\_level=10}}{}
Bases: \sphinxcode{\sphinxupquote{object}}

Project level system models:
\begin{itemize}
\item {} 
Assembles system configurations

\item {} 
Performs timestep simulation

\item {} 
Returns annual and timestep project and household
level results, such as gas and electricity use, heat delivered,
unmet demand and solar fraction.

\end{itemize}

Parameters:
\begin{quote}
\begin{description}
\item[{sys\_params: pd df}] \leavevmode
Main system component performance parameters per project
Default: None (default model parameters will get used)

\item[{backup\_params: pd df}] \leavevmode
Backup system performance parameters per project.
It should contain a household ID column, otherwise
columns identical to params.

\item[{sys\_sizes: pd df}] \leavevmode
Main system component sizes
Default: 1. (see individual components for specifics)

\item[{backup\_sizes: pd df}] \leavevmode
Backup system component sizes, contains household id column
Default: 1. (see individual components for specifics)

\item[{weather: pd df}] \leavevmode
Weather data timeseries.
Number of rows equals the number of timesteps. Can be
generated using the Source.irradiation\_and\_water\_main
method

Example:

\begin{sphinxVerbatim}[commandchars=\\\{\}]
\PYG{g+gp}{\PYGZgt{}\PYGZgt{}\PYGZgt{} }\PYG{n}{sourceASource}\PYG{p}{(}\PYG{n}{read\PYGZus{}from\PYGZus{}input\PYGZus{}dataframes} \PYG{o}{=} \PYG{n}{inputs}\PYG{p}{)}
\end{sphinxVerbatim}

Oakland climate zone in CEC weather data is ‘03’:

\begin{sphinxVerbatim}[commandchars=\\\{\}]
\PYG{g+gp}{\PYGZgt{}\PYGZgt{}\PYGZgt{} }\PYG{n+nb+bp}{self}\PYG{o}{.}\PYG{n}{weather} \PYG{o}{=} \PYG{n}{source}\PYG{o}{.}\PYG{n}{irradiation\PYGZus{}and\PYGZus{}water\PYGZus{}main}\PYG{p}{(}\PYG{l+s+s1}{\PYGZsq{}}\PYG{l+s+s1}{03}\PYG{l+s+s1}{\PYGZsq{}}\PYG{p}{,} \PYG{n}{method}\PYG{o}{=}\PYG{l+s+s1}{\PYGZsq{}}\PYG{l+s+s1}{isotropic diffuse}\PYG{l+s+s1}{\PYGZsq{}}\PYG{p}{)}
\end{sphinxVerbatim}

\item[{loads: pd df}] \leavevmode
A dataframe with loads for all individual household
served by the project level system. It should
contain 3 columns: household id, occupancy and a column
with a load array in m3 for each household.

Example:

\begin{sphinxVerbatim}[commandchars=\\\{\}]
\PYG{g+gp}{\PYGZgt{}\PYGZgt{}\PYGZgt{} }\PYG{n}{loads\PYGZus{}com} \PYG{o}{=} \PYG{n}{pd}\PYG{o}{.}\PYG{n}{DataFrame}\PYG{p}{(}\PYG{n}{data} \PYG{o}{=} \PYG{p}{[}\PYG{p}{[}\PYG{l+m+mi}{1}\PYG{p}{,} \PYG{n}{occ\PYGZus{}indiv} \PYG{o}{\PYGZhy{}} \PYG{l+m+mf}{1.}\PYG{p}{,} \PYG{l+m+mf}{0.8} \PYG{o}{*} \PYG{n}{load\PYGZus{}array}\PYG{p}{]}\PYG{p}{,}\PYG{p}{[}\PYG{l+m+mi}{2}\PYG{p}{,} \PYG{n}{occ\PYGZus{}indiv}\PYG{p}{,} \PYG{l+m+mf}{1.} \PYG{o}{*} \PYG{n}{load\PYGZus{}array}\PYG{p}{]}\PYG{p}{,}\PYG{p}{[}\PYG{l+m+mi}{3}\PYG{p}{,} \PYG{n}{occ\PYGZus{}indiv}\PYG{p}{,} \PYG{l+m+mf}{1.2} \PYG{o}{*} \PYG{n}{load\PYGZus{}array}\PYG{p}{]}\PYG{p}{,}\PYG{p}{[}\PYG{l+m+mi}{4}\PYG{p}{,} \PYG{n}{occ\PYGZus{}indiv} \PYG{o}{+} \PYG{l+m+mf}{1.}\PYG{p}{,} \PYG{l+m+mf}{1.4} \PYG{o}{*} \PYG{n}{load\PYGZus{}array}\PYG{p}{]}\PYG{p}{]}\PYG{p}{,}                    \PYG{n}{columns} \PYG{o}{=} \PYG{p}{[}\PYG{n+nb+bp}{self}\PYG{o}{.}\PYG{n}{c}\PYG{p}{[}\PYG{l+s+s1}{\PYGZsq{}}\PYG{l+s+s1}{id}\PYG{l+s+s1}{\PYGZsq{}}\PYG{p}{]}\PYG{p}{,} \PYG{n+nb+bp}{self}\PYG{o}{.}\PYG{n}{c}\PYG{p}{[}\PYG{l+s+s1}{\PYGZsq{}}\PYG{l+s+s1}{occ}\PYG{l+s+s1}{\PYGZsq{}}\PYG{p}{]}\PYG{p}{,} \PYG{n+nb+bp}{self}\PYG{o}{.}\PYG{n}{c}\PYG{p}{[}\PYG{l+s+s1}{\PYGZsq{}}\PYG{l+s+s1}{load\PYGZus{}m3}\PYG{l+s+s1}{\PYGZsq{}}\PYG{p}{]}\PYG{p}{]}\PYG{p}{)}
\end{sphinxVerbatim}

\item[{timestep: float, h}] \leavevmode
Duration of a single timestep, in hours
Default: 1. h

\item[{log\_level: None or python logger logging level,}] \leavevmode
Default: logging.DEBUG
This applies for a subset of the class functionality, mostly
used to deprecate logger messages for certain calculations.
For Example: log\_level = logging.ERROR will only throw error
messages and ignore INFO, DEBUG and WARNING.

\end{description}
\end{quote}

Examples:
\begin{quote}

See \sphinxcode{\sphinxupquote{{}`swh.system.tests.test\_models{}`}} module and
\sphinxcode{\sphinxupquote{{}`scripts/Project Level SWH System Tool.ipynb{}`}}
for examples on how to set up models for simulation.
\end{quote}
\index{conventional\_gas\_tank() (mswh.system.models.System method)@\spxentry{conventional\_gas\_tank()}\spxextra{mswh.system.models.System method}}

\begin{fulllineitems}
\phantomsection\label{\detokenize{source/mswh.system:mswh.system.models.System.conventional_gas_tank}}\pysiglinewithargsret{\sphinxbfcode{\sphinxupquote{conventional\_gas\_tank}}}{}{}
Basecase conventional gas tank water heater.
Make sure to size the tank according to the recommended
sizing rules, since the WHAM model does not apply to
tanks that are not appropriately sized.

Returns:
\begin{quote}
\begin{description}
\item[{ts\_proj: dict of arrays, W}] \leavevmode
Heat:
self.r{[}‘q\_del’{]}: delivered
self.r{[}‘gas\_use’{]}: gas consumed

\end{description}
\end{quote}

\end{fulllineitems}

\index{simulate() (mswh.system.models.System method)@\spxentry{simulate()}\spxextra{mswh.system.models.System method}}

\begin{fulllineitems}
\phantomsection\label{\detokenize{source/mswh.system:mswh.system.models.System.simulate}}\pysiglinewithargsret{\sphinxbfcode{\sphinxupquote{simulate}}}{\emph{type='gas\_tank\_wh'}}{}
Runs a 8760. hourly simulation of the
provided system type.

Parameters
\begin{quote}
\begin{description}
\item[{type: string}] \leavevmode
gas\_tank\_wh
solar\_thermal\_retrofit (gas tank backup at each household)
solar\_thermal\_new (gas tankless backup at each household)
solar\_electric

\end{description}
\end{quote}

Returns
\begin{quote}
\begin{description}
\item[{en\_use: dict}] \leavevmode
Total energy use for the analysis period:
‘gas’, Wh
‘electricity’, Wh

\item[{sys\_res: list}] \leavevmode
List containing detailed system level
output. See dedicated methods for details

\end{description}
\end{quote}

\end{fulllineitems}

\index{solar\_electric() (mswh.system.models.System method)@\spxentry{solar\_electric()}\spxextra{mswh.system.models.System method}}

\begin{fulllineitems}
\phantomsection\label{\detokenize{source/mswh.system:mswh.system.models.System.solar_electric}}\pysiglinewithargsret{\sphinxbfcode{\sphinxupquote{solar\_electric}}}{\emph{backup='electric'}}{}
Connects the components of the
solar electric system and enables simulation.

Parameters:
\begin{quote}
\begin{description}
\item[{backup: string}] \leavevmode
electric - instantaneous WHs (new installations)

\end{description}
\end{quote}

Returns:
\begin{quote}
\begin{description}
\item[{sys\_en\_use: dict}] \leavevmode
System level energy use for the analysis period:
‘electricity’, Wh

\item[{sol\_fra: dict}] \leavevmode
Solar fraction. Keys: ‘annual’, ‘monthly’

\item[{ts\_res: pd df}] \leavevmode
Colums populated with state variable timeseries, such as
average timstep heat rates and temperatures

\item[{res: dict}] \leavevmode
Summarizes ts\_res. Any heat rates are summed, while the
temperatures are averaged for the analysis period
(usually one year)

\item[{el\_use: dict}] \leavevmode
Electricity use broken into end uses
‘dist\_pump’ - distribution pump, if present

\item[{rel\_err: float}] \leavevmode
Balancing error due to limitations of finite
timestep averaging. More precisely, due to
selecting minimum tank temperature
as the lower between the water main and the ambient.

\end{description}
\end{quote}

\end{fulllineitems}

\index{solar\_thermal() (mswh.system.models.System method)@\spxentry{solar\_thermal()}\spxextra{mswh.system.models.System method}}

\begin{fulllineitems}
\phantomsection\label{\detokenize{source/mswh.system:mswh.system.models.System.solar_thermal}}\pysiglinewithargsret{\sphinxbfcode{\sphinxupquote{solar\_thermal}}}{\emph{backup='gas'}}{}
Connects the components of the solar thermal system and
simulates it in discrete timesteps.

Parameters:
\begin{quote}
\begin{description}
\item[{backup: string}] \leavevmode
retrofit - pulls from the basecase for each household
gas, electric - instantaneous WHs (new installations)

\end{description}
\end{quote}

Returns:
\begin{quote}
\begin{description}
\item[{self.cons\_total: pd df}] \leavevmode
Consumer level energy use {[}W{]}, heat rates {[}W{]},
average temperatures {[}K{]},
and solar fraction for the analysis period.

\item[{proj\_total: pd series}] \leavevmode
Project level energy use {[}W{]}, heat rates {[}W{]},
average temperatures {[}K{]},
and solar fraction for the analysis period.

\item[{sol\_fra: dict}] \leavevmode
Solar fraction. Keys: ‘annual’, ‘monthly’

\item[{pump\_el\_use: dict}] \leavevmode
Electricity use broken into end uses
‘dist\_pump’ - distribution pump, if present
‘sol\_pump’ - solar pump

\item[{ts\_res: pd df}] \leavevmode
Timestep project level results for all energy uses {[}W{]},
heat rates {[}W{]}, temperatures {[}K{]}, and the load.

\item[{backup\_ts\_cons: dict of dicts}] \leavevmode
Timestep household level results for energy uses {[}W{]},
and heat rates {[}W{]}.

\item[{rel\_err: float}] \leavevmode
Balancing error due to limitations of finite
timestep averaging.

\end{description}
\end{quote}

\end{fulllineitems}

\index{weather (mswh.system.models.System attribute)@\spxentry{weather}\spxextra{mswh.system.models.System attribute}}

\begin{fulllineitems}
\phantomsection\label{\detokenize{source/mswh.system:mswh.system.models.System.weather}}\pysigline{\sphinxbfcode{\sphinxupquote{weather}}}
\end{fulllineitems}


\end{fulllineitems}



\subsubsection{mswh.system.source\_and\_sink module}
\label{\detokenize{source/mswh.system:module-mswh.system.source_and_sink}}\label{\detokenize{source/mswh.system:mswh-system-source-and-sink-module}}\index{mswh.system.source\_and\_sink (module)@\spxentry{mswh.system.source\_and\_sink}\spxextra{module}}\index{SourceAndSink (class in mswh.system.source\_and\_sink)@\spxentry{SourceAndSink}\spxextra{class in mswh.system.source\_and\_sink}}

\begin{fulllineitems}
\phantomsection\label{\detokenize{source/mswh.system:mswh.system.source_and_sink.SourceAndSink}}\pysiglinewithargsret{\sphinxbfcode{\sphinxupquote{class }}\sphinxcode{\sphinxupquote{mswh.system.source\_and\_sink.}}\sphinxbfcode{\sphinxupquote{SourceAndSink}}}{\emph{input\_dfs=None}, \emph{random\_state=123}}{}
Bases: \sphinxcode{\sphinxupquote{object}}

Generates timeseries that are inputs to the simulation
model and are known prior to the simulation, such as
outdoor air temperature and end use load profiles.

Parameters:
\begin{quote}

input\_dfs : a dict of pd dfs
\begin{quote}

Dictionary of input dataframes
as read in from the input db by the Sql class
(see example in
\sphinxtitleref{test\_source\_and\_sink.SourceAndSinkTests.setUp})
\end{quote}

random\_state : numpy random state object or an integer
\begin{quote}

numpy random state object : if there is a need
to maintain the same random seed throughout the
analysis.

integer : a new random state object gets
instanteated at init
\end{quote}
\end{quote}
\index{demand\_estimate() (mswh.system.source\_and\_sink.SourceAndSink static method)@\spxentry{demand\_estimate()}\spxextra{mswh.system.source\_and\_sink.SourceAndSink static method}}

\begin{fulllineitems}
\phantomsection\label{\detokenize{source/mswh.system:mswh.system.source_and_sink.SourceAndSink.demand_estimate}}\pysiglinewithargsret{\sphinxbfcode{\sphinxupquote{static }}\sphinxbfcode{\sphinxupquote{demand\_estimate}}}{\emph{occ}}{}
Estimates gal/day demand as provided in the
CSI-Thermal Program Handbook, April 2016 for
installations with a known occupancy

Parameters:
\begin{quote}
\begin{description}
\item[{occ}] \leavevmode{[}float{]}
Number of individual household occupants

\end{description}
\end{quote}

\end{fulllineitems}

\index{irradiation\_and\_water\_main() (mswh.system.source\_and\_sink.SourceAndSink method)@\spxentry{irradiation\_and\_water\_main()}\spxextra{mswh.system.source\_and\_sink.SourceAndSink method}}

\begin{fulllineitems}
\phantomsection\label{\detokenize{source/mswh.system:mswh.system.source_and_sink.SourceAndSink.irradiation_and_water_main}}\pysiglinewithargsret{\sphinxbfcode{\sphinxupquote{irradiation\_and\_water\_main}}}{\emph{climate\_zone}, \emph{collector\_tilt='latitude'}, \emph{tilt\_standard\_deviation=None}, \emph{collector\_azimuth=0.0}, \emph{azimuth\_standard\_deviation=None}, \emph{location\_ground\_reflectance=0.16}, \emph{solar\_constant\_Wm2=1367.0}, \emph{method='isotropic diffuse'}, \emph{weather\_data\_source='cec'}, \emph{single\_row\_with\_arrays=False}}{}
Calculates the hourly total incident radiation on a tilted surface
for any climate zone in California. If weather data from the provided
database are passed as \sphinxtitleref{input\_dfs}, the user can specify a single
climate.

Two separate methods are available for use, with all equations
(along with the equation numbers provided in comments) as provided in
J. A. Duffie and W. A. Beckman, Solar engineering of thermal
processes, 3rd ed. Hoboken, N.J: Wiley, 2006.

Parameters:
\begin{quote}
\begin{description}
\item[{climate\_zone: string}] \leavevmode
String of two digits to indicate the CEC climate zone
being analyzed (‘01’ to ‘16’).

\item[{collector\_azimuth: float, default: 0.}] \leavevmode
The deviation of the projection on a horizontal
plane of the normal to the collector surface from
the local meridian, in degrees.  Allowable values
are between +/- 180 degrees (inclusive).  0 degrees
corresponds to due south, east is negative, and west
is positive.  Default value is 0 degrees (due south).

\item[{azimuth\_standard\_deviation: float, default: ‘None’}] \leavevmode
Final collector azimuth is a value drawn using a normal
distribution around the collector\_azimuth value
with a azimuth\_standard\_deviation standard deviation.
If set to ‘None’ the final collector azimuth
equals collector\_azimuth

\item[{collector\_tilt: float, default: ‘latitude’}] \leavevmode
The angle between the plane of the collector and the
horizontal, in degrees.  Allowable values are between
0 and 180 degrees (inclusive), and values greater than
90 degrees mean that the surface has a downward-facing
component. If a default flag is left unchanged, the code
will assign latitude value to the tilt as a good
approximation of a design collector or PV tilt.

\item[{tilt\_standard\_deviation: float, default: ‘None’}] \leavevmode
Final collector tilt is a value drawn using a normal
distribution around the collector\_tilt value
with a tilt\_standard\_deviation standard deviation.
If set to ‘None’ the final collector tilt
equals collector\_tilt

\item[{location\_ground\_reflectance: float, default: 0.16}] \leavevmode
The degree of ground reflectance.  Allowable
values are 0-1 (inclusive), with 0 meaning
no reflectance and 1 meaning very high
reflectance.  For reference, fresh snow has
a high ground reflectance of \textasciitilde{} 0.7.  Default
value is 0.16, which is the annual average surface
albedo averaged across the 16 CEC climate zones.

\item[{method: string, default: ‘HDKR anisotropic sky’}] \leavevmode
Calculation method to use for estimating the total irradiance
on the tilted collector surface. See notes below. Default
value is ‘HDKR anisotropic sky.’

\item[{solar\_constant\_Wm2: float, default: 1367.}] \leavevmode
Energy from the sun per unit time received on a unit
area of surface perpendicular to the direction of
propagation of the radiation at mean earth-sun distance
outside the atmosphere.  Default value is 1367 W/m\textasciicircum{}2.

\item[{weather\_data\_source: string, default: ‘cec’}] \leavevmode
The type of weather data being used to analyze the
climate zone for solar insolation.  Allowable values
are ‘cec’ and ‘tmy3.’  Default value is ‘cec.’

\item[{single\_row\_with\_arrays}] \leavevmode{[}boolean{]}
A flag to reformat the resulting dataframe in a row
of data where each resulting 8760 is stored as an
array

\end{description}
\end{quote}

Returns:
\begin{quote}
\begin{description}
\item[{data: pd df}] \leavevmode
Weather data frame with appended columns:
‘global\_tilt\_radiation\_Wm2’, ‘water\_main\_t\_F’,
‘water\_main\_t\_C’, ‘dry\_bulb\_C’, ‘wet\_bulb\_C’, ‘Tilt’,
‘Azimuth’{]}

\end{description}
\end{quote}

Notes:
\begin{quote}

The user can select one of two methods to use for
this calculation:
\begin{enumerate}
\def\theenumi{\arabic{enumi}}
\def\labelenumi{\theenumi )}
\makeatletter\def\p@enumii{\p@enumi \theenumi )}\makeatother
\item {} \begin{description}
\item[{‘isotropic diffuse’:}] \leavevmode
This model was derived by Liu and Jordan (1963).
All diffuse radiation is assumed to be isotropic.
It is the simpler and more conservative model,
and it has been widely used.

\end{description}

\item {} \begin{description}
\item[{‘HDKR anisotropic sky’:}] \leavevmode
This model combined methods from Hay and Davies (1980),
Klucher (1979), and Reindl, et al. (1990).
Diffuse radiation in this model is represented in two
parts: isotropic and circumsolar. The model also accounts
for horizon brightening.
This is also a simple model, but it has been found to be
slightly more accurate (and less conservative) than the
‘isotropic diffuse’ model.  For collectors tilted toward
the equator, this model is suggested.

\end{description}

\end{enumerate}
\end{quote}

\end{fulllineitems}


\end{fulllineitems}



\subsection{Tools}
\label{\detokenize{source/mswh.tools:tools}}\label{\detokenize{source/mswh.tools::doc}}

\subsubsection{mswh.tools.plots module}
\label{\detokenize{source/mswh.tools:module-mswh.tools.plots}}\label{\detokenize{source/mswh.tools:mswh-tools-plots-module}}\index{mswh.tools.plots (module)@\spxentry{mswh.tools.plots}\spxextra{module}}\index{Plot (class in mswh.tools.plots)@\spxentry{Plot}\spxextra{class in mswh.tools.plots}}

\begin{fulllineitems}
\phantomsection\label{\detokenize{source/mswh.tools:mswh.tools.plots.Plot}}\pysiglinewithargsret{\sphinxbfcode{\sphinxupquote{class }}\sphinxcode{\sphinxupquote{mswh.tools.plots.}}\sphinxbfcode{\sphinxupquote{Plot}}}{\emph{title=''}, \emph{label\_h='Time {[}h{]}'}, \emph{label\_v='Component performance'}, \emph{data\_headers=None}, \emph{save\_image=True}, \emph{legend=True}, \emph{outpath=''}, \emph{duration\_curve=False}, \emph{boxmode='group'}, \emph{notebook\_mode=False}, \emph{width=1200}, \emph{height=800}, \emph{fontsize=18}, \emph{legend\_x=0.4}, \emph{legend\_y=1.0}, \emph{margin\_l=200.0}, \emph{margin\_b=200.0}}{}
Bases: \sphinxcode{\sphinxupquote{object}}

Creates and saves plots to visualize and
correlate arrays, usually timeseries

Parameters:
\begin{quote}
\begin{description}
\item[{title: str}] \leavevmode
Plot title

\item[{data\_headers: list}] \leavevmode
A list of labels in the same order as the corresponding data
If None the labels will be the df column labels, or integer
indices if a list got provided

\item[{label\_h: str}] \leavevmode
Horizontal axis label

\item[{label\_v: str}] \leavevmode
Vertical axis label

\item[{legend: boolean}] \leavevmode
Plot the legend or not

\item[{save\_image: boolean}] \leavevmode
If True saves the created image with either
a given or default path and filename. Supported
file types are ‘png’ and ‘pdf’, as specified in
the filename.

\item[{duration\_curve: boolean}] \leavevmode
If True it sorts the columns (df or arrays)
and plots the duration\_curve, returns a
duration\_curve metric as a real

\item[{outpath: string or ‘’ (for current directory)}] \leavevmode
Path to save the png image of the plot

\end{description}

boxmean: True, False, ‘sd’, ‘Only Mean’
\begin{description}
\item[{notebook\_mode: boolean}] \leavevmode
Plot in the notebook if True

\item[{width: int}] \leavevmode
Image width

\item[{height: int}] \leavevmode
Image height

\item[{fontsize: int}] \leavevmode
Axis label font size

\end{description}
\end{quote}

Returns:
\begin{quote}
\begin{description}
\item[{fig: plotly figure if self.interactive}] \leavevmode
else True

\end{description}
\end{quote}
\index{box() (mswh.tools.plots.Plot method)@\spxentry{box()}\spxextra{mswh.tools.plots.Plot method}}

\begin{fulllineitems}
\phantomsection\label{\detokenize{source/mswh.tools:mswh.tools.plots.Plot.box}}\pysiglinewithargsret{\sphinxbfcode{\sphinxupquote{box}}}{\emph{dfs, plot\_cols=None, groupby\_cols=None, df\_cat=None, outfile='box.png', boxmean=False, colors={[}'\#3D9970', '\#FF4136', '\#FF851B'{]}, title='Energy Use', boxpoints='outliers'}}{}
Creates box plots for the chosen plot\_col and can
group plots by the groupby\_col.

Parameters:
\begin{quote}

dfs: list of dfs
\begin{description}
\item[{df\_cat: list of str}] \leavevmode
Indicator of the category carried by the dfs
(E.g. the dfs differ by housing type)

\item[{plot\_col: list of columns to plot, one from each df in}] \leavevmode
dfs. If multiple dfs are passed, the values will be
shown as groups on the plot

\item[{groupby\_cols: list of cols to use as x axis, from each}] \leavevmode
df. Use the same column if it has the same elements.
Use None if x axis category not used

\item[{boxpoints: False, ‘all’, ‘outliers’, ‘suspectedoutliers’}] \leavevmode
See \sphinxurl{https://plot.ly/python/reference/\#box}

\end{description}
\end{quote}

Returns:
\begin{quote}
\begin{description}
\item[{fig: plotly figure if self.interactive}] \leavevmode
else True

\end{description}
\end{quote}

\end{fulllineitems}

\index{scatter() (mswh.tools.plots.Plot method)@\spxentry{scatter()}\spxextra{mswh.tools.plots.Plot method}}

\begin{fulllineitems}
\phantomsection\label{\detokenize{source/mswh.tools:mswh.tools.plots.Plot.scatter}}\pysiglinewithargsret{\sphinxbfcode{\sphinxupquote{scatter}}}{\emph{data}, \emph{outfile='scatter.png'}, \emph{modes='lines+markers'}}{}
Creates a scatter plot

Parameters:
\begin{quote}
\begin{description}
\item[{data: array/list, pd series, list of arrays/lists, pd df}] \leavevmode
Provide a list or arrays/lists or a pandas dataframe.
The variables should be ordered in pairs such that
each odd variable in the list/first column in the df
gets assigned to the horizontal axis, each even
variable to the vertical axes. Each pair needs
to have the same length, but pairs can be of
a different length.

\item[{outfile: str}] \leavevmode
Filename, include .png, .png .pdf

\item[{modes: str or list of str}] \leavevmode
‘markers’, ‘lines’, ‘lines + markers’ or
a list of the above to assign to each plot
(one string in a list for each pair of data)

\end{description}
\end{quote}

Returns:
\begin{quote}
\begin{description}
\item[{fig: plotly figure if self.interactive}] \leavevmode
else True

\end{description}
\end{quote}

\end{fulllineitems}

\index{series() (mswh.tools.plots.Plot method)@\spxentry{series()}\spxextra{mswh.tools.plots.Plot method}}

\begin{fulllineitems}
\phantomsection\label{\detokenize{source/mswh.tools:mswh.tools.plots.Plot.series}}\pysiglinewithargsret{\sphinxbfcode{\sphinxupquote{series}}}{\emph{data}, \emph{index\_in\_a\_column=None}, \emph{outfile='series.png'}, \emph{modes='lines+markers'}}{}
Plots all series data against either the index or the first
provided series. It can sort the data and plot the duration\_curve.

Parameters:
\begin{quote}
\begin{description}
\item[{data: array/list, pd series, list of arrays/lists, pd df}] \leavevmode
Provide an array or a list if plotting a single
variable. If plotting multiple variables provide
a list of arrays or a pandas dataframe.
Horizontal axis corresponds to:
\begin{itemize}
\item {} 
if pd df: the index of the dataframe or the first columns of the dataframe

\item {} 
if list or arrays/lists: a range or array lenth or the first array/list in the list

\end{itemize}

All arrays in the list need to have the same length.

\item[{index\_in\_a\_column: boolean}] \leavevmode
Horizontal axis labels
If None, dataframe index is used, otherwise pass a
column label for a column (it will not be considered
as a series to plot)

\item[{outfile: str}] \leavevmode
Filename, include .png, .png .pdf

\item[{modes: str or list of str}] \leavevmode
‘markers’, ‘lines’, ‘lines+markers’ or
a list of the above to assign to each column
of data, excluding the first column if
index\_in\_a\_column is not None

\end{description}
\end{quote}

Returns:
\begin{quote}
\begin{description}
\item[{fig: plotly figure if self.interactive}] \leavevmode
else True

\end{description}
\end{quote}

\end{fulllineitems}


\end{fulllineitems}



\subsubsection{mswh.tools.unit\_converters module}
\label{\detokenize{source/mswh.tools:module-mswh.tools.unit_converters}}\label{\detokenize{source/mswh.tools:mswh-tools-unit-converters-module}}\index{mswh.tools.unit\_converters (module)@\spxentry{mswh.tools.unit\_converters}\spxextra{module}}\index{UnitConv (class in mswh.tools.unit\_converters)@\spxentry{UnitConv}\spxextra{class in mswh.tools.unit\_converters}}

\begin{fulllineitems}
\phantomsection\label{\detokenize{source/mswh.tools:mswh.tools.unit_converters.UnitConv}}\pysiglinewithargsret{\sphinxbfcode{\sphinxupquote{class }}\sphinxcode{\sphinxupquote{mswh.tools.unit\_converters.}}\sphinxbfcode{\sphinxupquote{UnitConv}}}{\emph{x\_in}, \emph{scale\_in=1.0}, \emph{scale\_out=1.0}}{}
Bases: \sphinxcode{\sphinxupquote{object}}

Unit conversions using conversion parameters from
ASHRAE Fundamentals 2017.

Parameters:
\begin{description}
\item[{x\_in: float, array}] \leavevmode
Input value to be converted to a desired unit

\item[{scale\_in: str or 1.}] \leavevmode
Scale of the input value, options: ‘k’, ‘kilo’,
‘mega’, ‘million’, ‘M’, ‘MM’, ‘giga’, ‘G’, ‘tera’, ‘T’,
‘peta’, ‘P’, ‘mili’, ‘micro’.
Default: 1.

\item[{scale\_out: str or 1.}] \leavevmode
Scale of the input value, options: ‘k’, ‘kilo’,
‘mega’, ‘million’, ‘M’, ‘MM’, ‘giga’, ‘G’, ‘tera’, ‘T’,
‘peta’, ‘P’, ‘mili’, ‘m’, ‘micro’.
Default: 1.

\end{description}

Examples:

To convert temperature from degF to degC

\begin{sphinxVerbatim}[commandchars=\\\{\}]
\PYG{g+gp}{\PYGZgt{}\PYGZgt{}\PYGZgt{} }\PYG{n}{t\PYGZus{}in\PYGZus{}degC} \PYG{o}{=} \PYG{n}{UnitConv}\PYG{p}{(}\PYG{n}{t\PYGZus{}in\PYGZus{}degF}\PYG{p}{)}\PYG{o}{.}\PYG{n}{degF\PYGZus{}degC}\PYG{p}{(}\PYG{n}{unit\PYGZus{}in}\PYG{o}{=}\PYG{l+s+s1}{\PYGZsq{}}\PYG{l+s+s1}{degF}\PYG{l+s+s1}{\PYGZsq{}}\PYG{p}{)}
\end{sphinxVerbatim}

To convert power in hp to kW:

\begin{sphinxVerbatim}[commandchars=\\\{\}]
\PYG{g+gp}{\PYGZgt{}\PYGZgt{}\PYGZgt{} }\PYG{n}{p\PYGZus{}in\PYGZus{}kW} \PYG{o}{=} \PYG{n}{UnitConv}\PYG{p}{(}\PYG{n}{p\PYGZus{}in\PYGZus{}hp}\PYG{p}{,} \PYG{n}{scale\PYGZus{}out}\PYG{o}{=}\PYG{l+s+s1}{\PYGZsq{}}\PYG{l+s+s1}{kilo}\PYG{l+s+s1}{\PYGZsq{}}\PYG{p}{)}\PYG{o}{.}\PYG{n}{hp\PYGZus{}W}\PYG{p}{(}\PYG{n}{unit\PYGZus{}in}\PYG{o}{=}\PYG{l+s+s1}{\PYGZsq{}}\PYG{l+s+s1}{hp}\PYG{l+s+s1}{\PYGZsq{}}\PYG{p}{)}
\end{sphinxVerbatim}

To convert energy from GJ to MMBtu:

\begin{sphinxVerbatim}[commandchars=\\\{\}]
\PYG{g+gp}{\PYGZgt{}\PYGZgt{}\PYGZgt{} }\PYG{n}{e\PYGZus{}MMBtu} \PYG{o}{=} \PYG{n}{UnitConv}\PYG{p}{(}\PYG{n}{e\PYGZus{}GJ}\PYG{p}{,} \PYG{n}{scale\PYGZus{}in}\PYG{o}{=}\PYG{l+s+s1}{\PYGZsq{}}\PYG{l+s+s1}{G}\PYG{l+s+s1}{\PYGZsq{}}\PYG{p}{,} \PYG{n}{scale\PYGZus{}out}\PYG{o}{=}\PYG{l+s+s1}{\PYGZsq{}}\PYG{l+s+s1}{MM}\PYG{l+s+s1}{\PYGZsq{}}\PYG{p}{)}\PYG{o}{.}\PYG{n}{Btu\PYGZus{}J}\PYG{p}{(}\PYG{n}{unit\PYGZus{}in}\PYG{o}{=}\PYG{l+s+s1}{\PYGZsq{}}\PYG{l+s+s1}{J}\PYG{l+s+s1}{\PYGZsq{}}\PYG{p}{)}
\end{sphinxVerbatim}
\index{Btu\_J() (mswh.tools.unit\_converters.UnitConv method)@\spxentry{Btu\_J()}\spxextra{mswh.tools.unit\_converters.UnitConv method}}

\begin{fulllineitems}
\phantomsection\label{\detokenize{source/mswh.tools:mswh.tools.unit_converters.UnitConv.Btu_J}}\pysiglinewithargsret{\sphinxbfcode{\sphinxupquote{Btu\_J}}}{\emph{unit\_in='Btu'}}{}
Converts work / energy / heat content between Btu and joule

Parameters:
\begin{quote}
\begin{description}
\item[{x: float, array}] \leavevmode
Input value

\item[{unit\_in: string, options: ‘Btu’, ‘J’}] \leavevmode
Unit of the input value

\end{description}
\end{quote}

Returns:
\begin{quote}
\begin{description}
\item[{x\_out: float, array}] \leavevmode
Output value

\end{description}
\end{quote}

\end{fulllineitems}

\index{Wh\_J() (mswh.tools.unit\_converters.UnitConv method)@\spxentry{Wh\_J()}\spxextra{mswh.tools.unit\_converters.UnitConv method}}

\begin{fulllineitems}
\phantomsection\label{\detokenize{source/mswh.tools:mswh.tools.unit_converters.UnitConv.Wh_J}}\pysiglinewithargsret{\sphinxbfcode{\sphinxupquote{Wh\_J}}}{\emph{unit\_in='J'}}{}
Converts work / energy / heat content between watthour and joule

Parameters:
\begin{quote}
\begin{description}
\item[{x: float, array}] \leavevmode
Input value

\item[{unit\_in: string, options: ‘Wh’, ‘J’}] \leavevmode
Unit of the input value

\end{description}
\end{quote}

Returns:
\begin{quote}
\begin{description}
\item[{x\_out: float, array}] \leavevmode
Output value

\end{description}
\end{quote}

\end{fulllineitems}

\index{degC\_K() (mswh.tools.unit\_converters.UnitConv method)@\spxentry{degC\_K()}\spxextra{mswh.tools.unit\_converters.UnitConv method}}

\begin{fulllineitems}
\phantomsection\label{\detokenize{source/mswh.tools:mswh.tools.unit_converters.UnitConv.degC_K}}\pysiglinewithargsret{\sphinxbfcode{\sphinxupquote{degC\_K}}}{\emph{unit\_in='degC'}}{}
Converts temperature between degree Celsius and Kelvin

Parameters:
\begin{quote}
\begin{description}
\item[{unit\_in: string, options: ‘K’, ‘degC’}] \leavevmode
Unit of the input value

\end{description}
\end{quote}

Returns:
\begin{quote}
\begin{description}
\item[{x\_out: float, array}] \leavevmode
Output value

\end{description}
\end{quote}

\end{fulllineitems}

\index{degF\_degC() (mswh.tools.unit\_converters.UnitConv method)@\spxentry{degF\_degC()}\spxextra{mswh.tools.unit\_converters.UnitConv method}}

\begin{fulllineitems}
\phantomsection\label{\detokenize{source/mswh.tools:mswh.tools.unit_converters.UnitConv.degF_degC}}\pysiglinewithargsret{\sphinxbfcode{\sphinxupquote{degF\_degC}}}{\emph{unit\_in='degF'}}{}
Converts temperature between degree Fahrenheit and Celsius

Parameters:
\begin{quote}
\begin{description}
\item[{unit\_in: string, options: ‘degF’, ‘degC’}] \leavevmode
Unit of the input value

\end{description}
\end{quote}

Returns:
\begin{quote}
\begin{description}
\item[{x\_out: float, array}] \leavevmode
Output value

\end{description}
\end{quote}

\end{fulllineitems}

\index{ft\_m() (mswh.tools.unit\_converters.UnitConv method)@\spxentry{ft\_m()}\spxextra{mswh.tools.unit\_converters.UnitConv method}}

\begin{fulllineitems}
\phantomsection\label{\detokenize{source/mswh.tools:mswh.tools.unit_converters.UnitConv.ft_m}}\pysiglinewithargsret{\sphinxbfcode{\sphinxupquote{ft\_m}}}{\emph{unit\_in='ft'}}{}
Converts length between foot
and meter

Parameters:
\begin{quote}
\begin{description}
\item[{x: float, array}] \leavevmode
Input value

\item[{unit\_in: string, options: ‘Wh’, ‘J’}] \leavevmode
Unit of the input value

\end{description}
\end{quote}

Returns:
\begin{quote}
\begin{description}
\item[{x\_out: float, array}] \leavevmode
Output value

\end{description}
\end{quote}

\end{fulllineitems}

\index{hp\_W() (mswh.tools.unit\_converters.UnitConv method)@\spxentry{hp\_W()}\spxextra{mswh.tools.unit\_converters.UnitConv method}}

\begin{fulllineitems}
\phantomsection\label{\detokenize{source/mswh.tools:mswh.tools.unit_converters.UnitConv.hp_W}}\pysiglinewithargsret{\sphinxbfcode{\sphinxupquote{hp\_W}}}{\emph{unit\_in='hp'}}{}
Converts power between watt and horsepower

Parameters:
\begin{quote}
\begin{description}
\item[{unit\_in: string, options: ‘hp’, ‘W’}] \leavevmode
Unit of the input value

\end{description}
\end{quote}

Returns:
\begin{quote}
\begin{description}
\item[{x\_out: float, array}] \leavevmode
Output value

\end{description}
\end{quote}

\end{fulllineitems}

\index{m3\_gal() (mswh.tools.unit\_converters.UnitConv method)@\spxentry{m3\_gal()}\spxextra{mswh.tools.unit\_converters.UnitConv method}}

\begin{fulllineitems}
\phantomsection\label{\detokenize{source/mswh.tools:mswh.tools.unit_converters.UnitConv.m3_gal}}\pysiglinewithargsret{\sphinxbfcode{\sphinxupquote{m3\_gal}}}{\emph{unit\_in='gal'}}{}
Converts volume between cubic meter and gallon

Parameters:
\begin{quote}
\begin{description}
\item[{unit\_in: string, options: ‘m3’, ‘gal’}] \leavevmode
Unit of the input value

\end{description}
\end{quote}

Returns:
\begin{quote}
\begin{description}
\item[{x\_out: float, array}] \leavevmode
Output value

\end{description}
\end{quote}

\end{fulllineitems}

\index{m3perh\_m3pers() (mswh.tools.unit\_converters.UnitConv method)@\spxentry{m3perh\_m3pers()}\spxextra{mswh.tools.unit\_converters.UnitConv method}}

\begin{fulllineitems}
\phantomsection\label{\detokenize{source/mswh.tools:mswh.tools.unit_converters.UnitConv.m3perh_m3pers}}\pysiglinewithargsret{\sphinxbfcode{\sphinxupquote{m3perh\_m3pers}}}{\emph{unit\_in='m3perh'}}{}
Converts volume flow between cubic meter
per hour and cubic meter per second

Parameters:
\begin{quote}
\begin{description}
\item[{x: float, array}] \leavevmode
Input value

\item[{unit\_in: string, options: ‘Wh’, ‘J’}] \leavevmode
Unit of the input value

\end{description}
\end{quote}

Returns:
\begin{quote}
\begin{description}
\item[{x\_out: float, array}] \leavevmode
Output value

\end{description}
\end{quote}

\end{fulllineitems}

\index{sqft\_m2() (mswh.tools.unit\_converters.UnitConv method)@\spxentry{sqft\_m2()}\spxextra{mswh.tools.unit\_converters.UnitConv method}}

\begin{fulllineitems}
\phantomsection\label{\detokenize{source/mswh.tools:mswh.tools.unit_converters.UnitConv.sqft_m2}}\pysiglinewithargsret{\sphinxbfcode{\sphinxupquote{sqft\_m2}}}{\emph{unit\_in='sqft'}}{}
Converts area between square foot
and square meter

Parameters:
\begin{quote}
\begin{description}
\item[{x: float, array}] \leavevmode
Input value

\item[{unit\_in: string, options: ‘Wh’, ‘J’}] \leavevmode
Unit of the input value

\end{description}
\end{quote}

Returns:
\begin{quote}
\begin{description}
\item[{x\_out: float, array}] \leavevmode
Output value

\end{description}
\end{quote}

\end{fulllineitems}

\index{therm\_J() (mswh.tools.unit\_converters.UnitConv method)@\spxentry{therm\_J()}\spxextra{mswh.tools.unit\_converters.UnitConv method}}

\begin{fulllineitems}
\phantomsection\label{\detokenize{source/mswh.tools:mswh.tools.unit_converters.UnitConv.therm_J}}\pysiglinewithargsret{\sphinxbfcode{\sphinxupquote{therm\_J}}}{\emph{unit\_in='therm'}}{}
Converts work / energy / heat content between therm and joule

Parameters:
\begin{quote}
\begin{description}
\item[{x: float, array}] \leavevmode
Input value

\item[{unit\_in: string, options: ‘therm’, ‘J’}] \leavevmode
Unit of the input value

\end{description}
\end{quote}

Returns:
\begin{quote}
\begin{description}
\item[{x\_out: float, array}] \leavevmode
Output value

\end{description}
\end{quote}

\end{fulllineitems}


\end{fulllineitems}

\index{Utility (class in mswh.tools.unit\_converters)@\spxentry{Utility}\spxextra{class in mswh.tools.unit\_converters}}

\begin{fulllineitems}
\phantomsection\label{\detokenize{source/mswh.tools:mswh.tools.unit_converters.Utility}}\pysiglinewithargsret{\sphinxbfcode{\sphinxupquote{class }}\sphinxcode{\sphinxupquote{mswh.tools.unit\_converters.}}\sphinxbfcode{\sphinxupquote{Utility}}}{\emph{quantity\_in}}{}
Bases: \sphinxcode{\sphinxupquote{object}}

Converts gas or electricity
consumption into commonly used units.

Parameters:
\begin{quote}
\begin{description}
\item[{quantity\_in: float, array}] \leavevmode
Quantity to be converted.
E.g. gas use in kJ

\end{description}
\end{quote}
\index{gas() (mswh.tools.unit\_converters.Utility method)@\spxentry{gas()}\spxextra{mswh.tools.unit\_converters.Utility method}}

\begin{fulllineitems}
\phantomsection\label{\detokenize{source/mswh.tools:mswh.tools.unit_converters.Utility.gas}}\pysiglinewithargsret{\sphinxbfcode{\sphinxupquote{gas}}}{\emph{unit\_in='kJ'}, \emph{unit\_out='MMBtu'}}{}
Converts gas consumption.

Parameters:
\begin{quote}
\begin{description}
\item[{unit\_in: string}] \leavevmode
Units of the input quantity that needs
to be converted.
Options: ‘kWh’, ‘kJ’

\item[{unit\_out: string}] \leavevmode
Desired output unit

\end{description}
\end{quote}

Returns:
\begin{quote}
\begin{description}
\item[{gas\_use: float}] \leavevmode
Gas use in output units

\end{description}
\end{quote}

\end{fulllineitems}


\end{fulllineitems}



\subsection{Database Communication}
\label{\detokenize{source/mswh.comm:database-communication}}\label{\detokenize{source/mswh.comm::doc}}

\subsubsection{mswh.comm.sql module}
\label{\detokenize{source/mswh.comm:module-mswh.comm.sql}}\label{\detokenize{source/mswh.comm:mswh-comm-sql-module}}\index{mswh.comm.sql (module)@\spxentry{mswh.comm.sql}\spxextra{module}}\index{Sql (class in mswh.comm.sql)@\spxentry{Sql}\spxextra{class in mswh.comm.sql}}

\begin{fulllineitems}
\phantomsection\label{\detokenize{source/mswh.comm:mswh.comm.sql.Sql}}\pysiglinewithargsret{\sphinxbfcode{\sphinxupquote{class }}\sphinxcode{\sphinxupquote{mswh.comm.sql.}}\sphinxbfcode{\sphinxupquote{Sql}}}{\emph{path\_OR\_dbconn}}{}
Bases: \sphinxcode{\sphinxupquote{object}}

Performs python-sqlite db communication.

Parameters:
\begin{quote}
\begin{description}
\item[{path\_OR\_dbconn: str or a database connection instance}] \leavevmode
Full path to a database file or an already
instantiated connection object

\end{description}
\end{quote}
\index{commit() (mswh.comm.sql.Sql method)@\spxentry{commit()}\spxextra{mswh.comm.sql.Sql method}}

\begin{fulllineitems}
\phantomsection\label{\detokenize{source/mswh.comm:mswh.comm.sql.Sql.commit}}\pysiglinewithargsret{\sphinxbfcode{\sphinxupquote{commit}}}{\emph{sql\_command}, \emph{close=False}}{}
Execute a custom sql command

Parameters:
\begin{quote}
\begin{description}
\item[{sql\_command: string}] \leavevmode
sql\_command to exectute

\end{description}
\end{quote}

Returns:
\begin{quote}
\begin{description}
\item[{close: boolean, default=False}] \leavevmode
If True, closes the connection to db

\end{description}
\end{quote}

\end{fulllineitems}

\index{csv2table() (mswh.comm.sql.Sql method)@\spxentry{csv2table()}\spxextra{mswh.comm.sql.Sql method}}

\begin{fulllineitems}
\phantomsection\label{\detokenize{source/mswh.comm:mswh.comm.sql.Sql.csv2table}}\pysiglinewithargsret{\sphinxbfcode{\sphinxupquote{csv2table}}}{\emph{path\_to\_csv}, \emph{table\_name}, \emph{column\_label\_row=0}, \emph{converters=None}, \emph{close=False}}{}
Use to update bulk price or performance data.
If same named table exists, it gets replaced

Parameters:
\begin{quote}
\begin{description}
\item[{path\_to\_csv: str}] \leavevmode
Full path to the csv table

\item[{table\_name: str}] \leavevmode
sql table name of choice

\item[{column\_label\_row: int, default=0}] \leavevmode
Index of the row which gets
converted into column labels

\item[{converters: dict, default=None}] \leavevmode
According to pandas documentation:
Dict of functions for converting
values in columns. Keys can be integers
or column labels.

\item[{close: boolean, default=False}] \leavevmode
If True, closes the connection to db

\end{description}
\end{quote}

\end{fulllineitems}

\index{pd2table() (mswh.comm.sql.Sql method)@\spxentry{pd2table()}\spxextra{mswh.comm.sql.Sql method}}

\begin{fulllineitems}
\phantomsection\label{\detokenize{source/mswh.comm:mswh.comm.sql.Sql.pd2table}}\pysiglinewithargsret{\sphinxbfcode{\sphinxupquote{pd2table}}}{\emph{df}, \emph{table\_name}, \emph{close=False}}{}
Write a dataframe out to the database.
If same named table exists, it gets replaced

Parameters:
\begin{quote}
\begin{description}
\item[{table\_name: str}] \leavevmode
sql table name

\item[{close: boolean, default=False}] \leavevmode
If True, closes the connection to db

\end{description}
\end{quote}

\end{fulllineitems}

\index{table2pd() (mswh.comm.sql.Sql method)@\spxentry{table2pd()}\spxextra{mswh.comm.sql.Sql method}}

\begin{fulllineitems}
\phantomsection\label{\detokenize{source/mswh.comm:mswh.comm.sql.Sql.table2pd}}\pysiglinewithargsret{\sphinxbfcode{\sphinxupquote{table2pd}}}{\emph{table\_name}, \emph{column\_label\_row=0}}{}
Reads in a single sql table.

Parameters:
\begin{quote}
\begin{description}
\item[{table\_name: str}] \leavevmode
sql table name

\item[{column\_label\_row: int, default=0}] \leavevmode
Index of the row which gets
converted into column labels

\end{description}
\end{quote}

Returns:
\begin{quote}
\begin{description}
\item[{df: pandas dataframe}] \leavevmode
Sql table read in as a pandas df.

\end{description}
\end{quote}

\end{fulllineitems}

\index{tables2dict() (mswh.comm.sql.Sql method)@\spxentry{tables2dict()}\spxextra{mswh.comm.sql.Sql method}}

\begin{fulllineitems}
\phantomsection\label{\detokenize{source/mswh.comm:mswh.comm.sql.Sql.tables2dict}}\pysiglinewithargsret{\sphinxbfcode{\sphinxupquote{tables2dict}}}{\emph{close=True}}{}
Reads all tables contained in a
sql database and converts them to a
pandas dataframe.

Parameters:
\begin{quote}
\begin{description}
\item[{close: boolean, default=True}] \leavevmode
If True, closes the connection to db

\end{description}
\end{quote}

Returns:
\begin{quote}
\begin{description}
\item[{data: dict of pandas dataframes}] \leavevmode
Saves each of the sql tabels as a
pandas dataframe under a sql table
name as a key

\end{description}
\end{quote}

\end{fulllineitems}


\end{fulllineitems}



\chapter{Copyright Notice}
\label{\detokenize{source/legal:copyright-notice}}\label{\detokenize{source/legal::doc}}
Multiscale Solar Water Heating (MSWH) Copyright (c) 2019, The
Regents of the University of California, through Lawrence Berkeley National
Laboratory (subject to receipt of any required approvals from the U.S.
Dept. of Energy).  All rights reserved.

If you have questions about your rights to use or distribute this software,
please contact Berkeley Lab’s Intellectual Property Office at
\sphinxhref{mailto:IPO@lbl.gov}{IPO@lbl.gov}.

NOTICE.  This Software was developed under funding from the U.S. Department
of Energy and the U.S. Government consequently retains certain rights.  As
such, the U.S. Government has been granted for itself and others acting on
its behalf a paid-up, nonexclusive, irrevocable, worldwide license in the
Software to reproduce, distribute copies to the public, prepare derivative
works, and perform publicly and display publicly, and to permit other to do
so.


\chapter{License Agreement}
\label{\detokenize{source/legal:license-agreement}}
Multiscale Solar Water Heating (MSWH) Copyright (c) 2019, The
Regents of the University of California, through Lawrence Berkeley National
Laboratory (subject to receipt of any required approvals from the U.S.
Dept. of Energy).  All rights reserved.

Redistribution and use in source and binary forms, with or without
modification, are permitted provided that the following conditions are met:

(1) Redistributions of source code must retain the above copyright notice,
this list of conditions and the following disclaimer.

(2) Redistributions in binary form must reproduce the above copyright
notice, this list of conditions and the following disclaimer in the
documentation and/or other materials provided with the distribution.

(3) Neither the name of the University of California, Lawrence Berkeley
National Laboratory, U.S. Dept. of Energy nor the names of its contributors
may be used to endorse or promote products derived from this software
without specific prior written permission.

THIS SOFTWARE IS PROVIDED BY THE COPYRIGHT HOLDERS AND CONTRIBUTORS “AS IS”
AND ANY EXPRESS OR IMPLIED WARRANTIES, INCLUDING, BUT NOT LIMITED TO, THE
IMPLIED WARRANTIES OF MERCHANTABILITY AND FITNESS FOR A PARTICULAR PURPOSE
ARE DISCLAIMED. IN NO EVENT SHALL THE COPYRIGHT OWNER OR CONTRIBUTORS BE
LIABLE FOR ANY DIRECT, INDIRECT, INCIDENTAL, SPECIAL, EXEMPLARY, OR
CONSEQUENTIAL DAMAGES (INCLUDING, BUT NOT LIMITED TO, PROCUREMENT OF
SUBSTITUTE GOODS OR SERVICES; LOSS OF USE, DATA, OR PROFITS; OR BUSINESS
INTERRUPTION) HOWEVER CAUSED AND ON ANY THEORY OF LIABILITY, WHETHER IN
CONTRACT, STRICT LIABILITY, OR TORT (INCLUDING NEGLIGENCE OR OTHERWISE)
ARISING IN ANY WAY OUT OF THE USE OF THIS SOFTWARE, EVEN IF ADVISED OF THE
POSSIBILITY OF SUCH DAMAGE.

You are under no obligation whatsoever to provide any bug fixes, patches,
or upgrades to the features, functionality or performance of the source
code (“Enhancements”) to anyone; however, if you choose to make your
Enhancements available either publicly, or directly to Lawrence Berkeley
National Laboratory, without imposing a separate written license agreement
for such Enhancements, then you hereby grant the following license: a
non-exclusive, royalty-free perpetual license to install, use, modify,
prepare derivative works, incorporate into other computer software,
distribute, and sublicense such enhancements or derivative works thereof,
in binary and source code form.


\renewcommand{\indexname}{Python Module Index}
\begin{sphinxtheindex}
\let\bigletter\sphinxstyleindexlettergroup
\bigletter{m}
\item\relax\sphinxstyleindexentry{mswh.comm.sql}\sphinxstyleindexpageref{source/mswh.comm:\detokenize{module-mswh.comm.sql}}
\item\relax\sphinxstyleindexentry{mswh.system.components}\sphinxstyleindexpageref{source/mswh.system:\detokenize{module-mswh.system.components}}
\item\relax\sphinxstyleindexentry{mswh.system.models}\sphinxstyleindexpageref{source/mswh.system:\detokenize{module-mswh.system.models}}
\item\relax\sphinxstyleindexentry{mswh.system.source\_and\_sink}\sphinxstyleindexpageref{source/mswh.system:\detokenize{module-mswh.system.source_and_sink}}
\item\relax\sphinxstyleindexentry{mswh.tools.plots}\sphinxstyleindexpageref{source/mswh.tools:\detokenize{module-mswh.tools.plots}}
\item\relax\sphinxstyleindexentry{mswh.tools.unit\_converters}\sphinxstyleindexpageref{source/mswh.tools:\detokenize{module-mswh.tools.unit_converters}}
\end{sphinxtheindex}

\renewcommand{\indexname}{Index}
\printindex
\end{document}